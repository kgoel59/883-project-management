\chapter{Project Risk Management}
\begin{longtable}{|p{1cm}|p{2.5cm}|c|p{1.8cm}|p{1.4cm}|p{1cm}|p{1.8cm}|p{2cm}|}
    \caption{Risk Management Table} 
    \label{tab:rmt} \\
    \hline
    \textbf{Risk ID} & \textbf{Risk Description} & \textbf{Type} & \textbf{Probability (1-10)} & \textbf{Impact (1-10)} & \textbf{Risk Score} & \textbf{Response Strategy} & \textbf{Cost Estimate} \\
    \hline
    \endfirsthead
    \caption[]{Risk Management Table (Continued)} \\
    \hline
    \textbf{Risk ID} & \textbf{Risk Description} & \textbf{Type} & \textbf{Probability (1-10)} & \textbf{Impact (1-10)} & \textbf{Risk Score} & \textbf{Response Strategy} & \textbf{Cost Estimate} \\
    \hline
    \endhead
    \hline
    \endfoot
    \hline
    \endlastfoot
    R1 & Key team members leaving the company & Negative & 8 & 9 & 72 & Mitigate & \$8500 \\
    \hline
    R2 & Uncooperative users & Negative & 7 & 8 & 56 & Mitigate & \$3500 \\
    \hline
    R3 & High employee engagement & Positive & 5 & 9 & 45 & Enhance & \$5000 + \$1000/month \\
    \hline
    R4 & Inability to track data effectively & Negative & 6 & 7 & 42 & Mitigate & \$4000 \\
    \hline
    R5 & Team members not providing good status information & Negative & 6 & 7 & 42 & Mitigate & \$2000 \\
    \hline
    R6 & Successful negotiation of lower health insurance premiums & Positive & 4 & 8 & 32 & Enhance & \$2000 \\
    \hline
\end{longtable}

\section*{Risk Actions}
\begin{longtable}{|p{1cm}|p{5cm}|p{9cm}|}
    \caption{Risk Action Table} \label{tab:title} \\
    \hline
    \textbf{Risk ID} & \textbf{Risk Description} & \textbf{Action} \\
    \hline
    \endfirsthead
    \caption[]{Risk Action Table (Continued)} \\
    \hline
    \textbf{Risk ID} & \textbf{Risk Description} & \textbf{Action} \\
    \hline
    \endhead
    \hline
    \endfoot
    \hline
    \endlastfoot
    R1 & Key team members leaving the company & Cross-train the members, update project plan, and implement knowledge transfer as well as team building exercises. \\
    \hline
    R2 & Uncooperative users & Conduct user engagement sessions regularly, gather feedback, and offer training sessions. \\
    \hline
    R3 & High employee engagement & Introduce incentive program, marketing campaign, and adjust program based on feedback. \\
    \hline
    R4 & Inability to track data effectively & Implement a robust data tracking system and provide training on how to use it. \\
    \hline
    R5 & Team members not providing good status information & Implement regular status meetings, develop a reporting template, and provide training on reporting. \\
    \hline
    R6 & Successful negotiation of lower health insurance premiums & Collect and present data on improved health metrics, engage in negotiations with insurance providers. \\
    \hline
\end{longtable}

\pagebreak
\section*{Risk Scores}
\begin{longtable}{|c|c|c|c|}
    \caption{Risk Score Table} \label{tab:title} \\
    \hline
    \textbf{Risk ID} & \textbf{Probability} & \textbf{Impact} & \textbf{Risk Score} \\
    \hline
    \endfirsthead
    \caption[]{Risk Score Table (Continued)} \\
    \hline
    \textbf{Risk ID} & \textbf{Probability} & \textbf{Impact} & \textbf{Risk Score} \\
    \hline
    \endhead
    \hline
    \endfoot
    \hline
    \endlastfoot
    R1 & 8 & 9 & 72 \\
    \hline
    R2 & 7 & 8 & 56 \\
    \hline
    R3 & 5 & 9 & 45 \\
    \hline
    R4 & 6 & 7 & 42 \\
    \hline
    R5 & 6 & 7 & 42 \\
    \hline
    R6 & 4 & 8 & 32 \\
    \hline
\end{longtable}

\section*{Rationale for Risk Scores}

\subsection*{Negative Risk: Key team members leaving the company (R1)}
\begin{itemize}
    \item \textbf{Probability: 8} - Given the history of the project and existing turnover, the probability of team members leaving is the highest. 
    \item \textbf{Impact: 9} - Losing key team members can significantly disrupt project progress, requiring time and resources to onboard new members and carry out missed work.
\end{itemize}

\subsection*{Positive Risk: High employee engagement in the programs leading to improved health (R3)}
\begin{itemize}
    \item \textbf{Probability: 5} - There's a moderate chance of high engagement if the programs are well-promoted and incentivized. 
    \item \textbf{Impact: 9} - If successful, high engagement can lead to significant improvements in employee health, thereby achieving project goals and reducing insurance premiums.
\end{itemize}

\section*{Response Strategies}

\subsection*{Negative Risk: Key team members leaving the company (R1)} 
\begin{outline}
    \1 \textbf{Response Strategy: Mitigate}
        \2 \textbf{Tasks:}
            \3 \textbf{Create a knowledge transfer plan:} Document all critical processes and knowledge. (2 weeks, \$2000)
            \3 \textbf{Cross-train team members:} Ensure multiple team members are knowledgeable about critical tasks. (4 weeks, \$4000)
            \3 \textbf{Develop a succession plan:} Identify potential internal replacements and start preliminary training. (2 weeks, \$1500)
            \3 \textbf{Engage in team-building activities:} Improve team cohesion and job satisfaction. (1 week, \$1000)
\end{outline}
\begin{outline}
    \1 \textbf{Time and Cost Estimates:} 
        \2 Total time: 9 weeks 
        \2 Total cost: \$8500
\end{outline}
\pagebreak
\subsection*{Positive Risk: High employee engagement in the programs leading to improved health (R3)}
\begin{outline}
    \1 \textbf{Response Strategy: Enhance} 
        \2 \textbf{Tasks:}
            \3 \textbf{Develop a marketing campaign:} Create materials to promote the programs. (3 weeks, \$3000)
            \3 \textbf{Introduce incentive programs:} Design rewards for participation and achievements. (2 weeks, \$2000)
            \3 \textbf{Regularly collect and share success stories:} Highlight positive outcomes to motivate others. (Ongoing, \$1000/month)
            \3 \textbf{Monitor and adjust programs:} Continuously improve based on feedback and participation rates. (Ongoing, \$1000/month)
    \1 \textbf{Time and Cost Estimates:} 
        \2 Initial time: 5 weeks 
        \2 Initial cost: \$5000 
        \2 Ongoing cost: \$1000/month

\end{outline}



\FloatBarrier
\newpage