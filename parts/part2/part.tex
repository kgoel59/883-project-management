\chapter{Project Scope Management}

In this part, we will detail the project scope, clearly defining what is included and what is excluded from the project. We will also describe the development of the Work Breakdown Structure (WBS) and the Gantt chart for this project. These tools are essential for ensuring that the project remains focused on its intended goals and is executed efficiently without unnecessary expansion or deviation. By establishing these parameters, we aim to provide a structured approach to managing the project's scope, facilitating better control over project deliverables and timelines.


\section{Stakeholder Identification}
The first step before developing the project scope is to identify all stakeholders. This process ensures that the needs and expectations of every individual or group affected by the project are considered from the outset.

The stakeholders identified for the project are listed in Table \ref{tab:stakeholders}.


\begin{table}[h]
\centering
\caption{Stakeholders in the Recreation and Wellness Project}
\label{tab:stakeholders}
\begin{tabular}{|m{4cm}|m{8cm}|}
\hline
\textbf{Stakeholder} & \textbf{Role/Interest} \\
\hline
Employees & Direct beneficiaries, interested in wellness programs and facilities \\
\hline
Project Management Team & Responsible for planning, executing, and closing the project \\
\hline
Human Resources & Interested in employee satisfaction and retention \\
\hline
IT Department & Responsible for supporting technology needs and system integration \\
\hline
Health and Safety Officers & Ensure compliance with health and safety regulations \\
\hline
External Vendors & Provide necessary equipment or services for the wellness programs \\
\hline
Senior Management & Strategic oversight and funding decisions \\
\hline
\end{tabular}
\end{table}


\section{Requirements Analysis}

The next step in developing the project scope is to gather and list the requirements necessary for project completion. In this section, we will detail the requirement gathering process and present the requirements in a Requirement Traceability Matrix (RTM). This matrix will help ensure that each requirement is clearly linked to project objectives and can be traced throughout the project lifecycle.


\subsection{Requirements Gathering}

To ensure a thorough evaluation of all system aspects relevant to the project, requirements have been collected from various sources using diverse methods:

\begin{enumerate}
    \item \textbf{Stakeholder Interviews:} In-depth interviews with identified stakeholders are conducted to obtain detailed insights into their specific requirements and expectations.
    \item \textbf{Employee Surveys:} Surveys are distributed among employees to gather a broad spectrum of data on their perspectives and needs.
    \item \textbf{Focus Groups:} Focus group discussions are organized to explore particular issues or topics of interest in greater depth with a select group of stakeholders.
    \item \textbf{Review of Existing Systems and Data:} Current systems and historical data are analyzed to establish a baseline and identify potential areas for improvement.
    \item \textbf{Benchmarking:} Investigated and compared existing wellness programs or applications in other organizations to identify best practices and innovative features that could be incorporated into our project.

\end{enumerate}

\subsection{Requirement Traceability Matrix}

Once the requirements have been collected, we constructed a Requirement Traceability Matrix (RTM) to map each requirement back to its source. This matrix serves as a critical tool for ensuring that all requirements are clearly linked to their origins and that they are fully addressed throughout the project lifecycle. 

Table \ref{tab:rtm} list the Requirement Traceability Matrix (RTM) with its source and links it to specific features.

\begin{longtable}{|m{1cm}|m{5cm}|m{3.5cm}|m{3.5cm}|}
\caption{Requirement Traceability Matrix} \label{tab:rtm} \\
\hline
\textbf{ID} & \textbf{Requirement} & \textbf{Source} & \textbf{Feature} \\
\hline
\endfirsthead

\multicolumn{4}{c}%
{{\tablename\ \thetable{} -- \textit{Continued from previous page}}} \\
\hline
\textbf{ID} & \textbf{Requirement} & \textbf{Source} & \textbf{Feature} \\
\hline
\endhead
\endfoot

\hline
\endlastfoot

R1 & Develop a user-friendly interface for the wellness program application. & Stakeholder Interviews & User Interface Design \\
\hline
R2 & Ensure compliance with data privacy laws in health management applications. & Legal Requirements & Data Security \\
\hline
R3 & Implement a system for tracking employee participation in wellness activities. & Employee Surveys & Activity Tracking \\
\hline
R4 & Integrate third-party services for mental health resources. & Focus Groups & Third-Party Integration \\
\hline
R5 & Create a feedback mechanism for users to report issues and suggestions. & Employee Surveys & User Feedback System \\
\hline
R6 & Enable customization of wellness programs to meet individual health goals. & Stakeholder Interviews & Personalization Settings \\
\end{longtable}

\section{Project Scope Statement - Version 1}

This section presents the initial version of the project scope statement as created by the project manager. The document outlines key project objectives, deliverables, and the overall approach.

\includepdf[pages=-]{parts/part2/scope.pdf}