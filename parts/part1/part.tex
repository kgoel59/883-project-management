\chapter{Project Integration Management}
Our first step in ensuring the successful delivery of this project for our client is to gain a comprehensive understanding of their primary goals and objectives. This part provides an overview of the main objectives, initial proposals, and business case studies that have been developed in consultation with the client.

Here is a refined version of the section on client case study:

\section{Client Case Study}

Manage Your Health, Inc. (MYH) is a global healthcare provider ranked among the Fortune 500. With over 25,000 employees worldwide, including 20,000 full-time and 5,000 part-time workers, MYH is committed to cutting internal costs, enhancing cross-selling of products, and leveraging new web technologies to foster collaboration among employees, customers, and suppliers, thereby enhancing the delivery of healthcare products and services.

MYH has identified three potential projects in line with their strategic objectives:

\begin{enumerate}[label=\textbf{\arabic*.}]
    \item \textbf{Health Coverage Costs Business Model:} This project involves developing a secure application to model and analyze healthcare expenses. It requires an initial investment of \$100,000 and is projected to save \$20 annually for each full-time employee.
    
    \item \textbf{Web-Enhanced Communications System:} The goal of this project is to implement a web-based system that streamlines the development and delivery of products. With a development cost of \$3 million, the system is expected to save \$2 million annually, notwithstanding ongoing maintenance costs.
    
    \item \textbf{Recreation and Wellness Intranet Project:} This initiative aims to launch an intranet application to enhance employee health and wellness, potentially reducing healthcare premiums and saving \$30 per employee each year through better health outcomes.
\end{enumerate}

MYH requests a preliminary analysis to determine which project best meets their strategic goals and asks for a detailed business plan for the chosen proposal.

\section{Project Analysis}

In response to the request from our client, we initiated a preliminary analysis of the proposed projects. 

\subsection{Objectives Analysis}

The analysis began with an understanding of the strategic goals of Manage Your Health, Inc. (MYH), which are summarized in Table \ref{tab:myh_goals} based on our discussions with the client.

\begin{table}[h]
    \centering
    \begin{tabular}{|c|p{10cm}|}
    \hline
    \textbf{Goal} & \textbf{Description} \\
    \hline
    Reduce-Cost & Aim to decrease internal costs to enhance efficiency and boost profitability. \\
    \hline
    Business-Growth & Increase market penetration and revenue through enhanced cross-selling opportunities. \\
    \hline
    Develop & Improve collaboration among employees, customers, and suppliers with new web-based technologies, thereby optimizing the development and delivery of healthcare products and services. \\
    \hline
    \end{tabular}
    \caption{Strategic Goals of Manage Your Health, Inc.}
    \label{tab:myh_goals}
\end{table}

\FloatBarrier

This structured approach allows us to align each project proposal with MYH's strategic objectives, facilitating an informed decision-making process for developing the subsequent business plan.

\subsection{Proposal Analysis}

Once the objectives are clear, our team conducted a thorough analysis of each proposal to ensure alignment with client goals. We focused on several key metrics and assigned different weights to each criteria.

\begin{itemize}
    \item Tie to business strategy - 10\%
    \item Upfront cost - 25\%
    \item Potential net savings - 25\%
    \item Realistic technology - 15\%
    \item In-house expertise - 10\%
    \item Potential resistance - 15\%
\end{itemize}
The results of our finding are summarized in Table \ref{tab:project_com}.

\begin{table}[h]
    \centering
    \begin{tabularx}{\textwidth}{>{\hsize=0.7\hsize}X>{\hsize=0.5\hsize}X>{\hsize=1.2\hsize}X>{\hsize=1.3\hsize}X>{\hsize=1.3\hsize}X}
    \toprule
    \textbf{Criteria} & \textbf{Weight} & \textbf{Health Coverage Costs Business Model} & \textbf{Web-Enhanced Communications System} & \textbf{Recreation and Wellness Intranet Project} \\
    \midrule
    Tie-to business strategy & 10\% & Reduce-Cost (1/3) & Business-Growth, Develop (2/3) & Reduce-Cost (1/3) \\
    Upfront cost & 25\% & \$100,000 & \$3,000,000 & \$200,000 \\
    Potential net savings & 25\% & \$1,600,000 & \$6,000,000 & \$2,400,000 \\
    Realistic technology & 15\% & Data Analysts needed to analyze the premiums of current and past employees linked to 10 different insurance companies. & The project is highly achievable with modern technology, as we all have experience with similar systems. & The project is highly achievable with modern technology, as we all have experience with similar systems. \\
    In-house expertise & 10\% & Organization might need to hire more experienced Data Analysts, even though we have expert developers to implement the application to analyze the data. & New components to be implemented and new services to be provided, such as delivery services, customer support, and suppliers management, necessitate hiring quite a few staff. & Project is easy to implement with proper guidance from a Team Lead. \\
    Potential resistance & 15\% & Project might not face much resistance since the application is relatively easy to build. & Project might not face much resistance since the application is relatively easy to build. & Senior employees might resist involvement in recreational programs due to greater responsibilities like family care. Other employees might also show disinterest. \\
    \bottomrule
    \end{tabularx}
    \caption{Comparison of Project Criteria and Their Impact}
    \label{tab:project_com}
\end{table}

\FloatBarrier

\subsection{Weighted Model}

To further analyze our findings, we quantified the results from Table \ref{tab:project_com} to create a weighted model, as shown in Table \ref{tab:project_com_2}.

\begin{table}[h]
    \centering
    \begin{tabularx}{\textwidth}{>{\hsize=0.7\hsize}X>{\hsize=0.5\hsize}X>{\hsize=1.2\hsize}X>{\hsize=1.3\hsize}X>{\hsize=1.3\hsize}X}
    \toprule
    \textbf{Criteria} & \textbf{Weight} & \textbf{Health Coverage Costs Business Model} & \textbf{Web-Enhanced Communications System} & \textbf{Recreation and Wellness Intranet Project} \\
    \midrule
    Tie to business strategy & 10\% & 50 & 70 & 60 \\
    Upfront cost & 25\% & 70 & 35 & 85 \\
    Potential net savings & 25\% & 70 & 30 & 90 \\
    Realistic technology & 15\% & 60 & 85 & 75 \\
    In-house expertise & 10\% & 55 & 40 & 80 \\
    Potential resistance & 15\% & 80 & 85 & 60 \\
    \midrule
    \textbf{Total} & 100\% & 66.5 & 52.75 & 78 \\
    \bottomrule
    \end{tabularx}
    \caption{Comparison of Project Criteria and Their Impact (Weighted Model)}
    \label{tab:project_com_2}
\end{table}

\begin{figure}[ht]
    \includegraphics[width=\textwidth]{images/wsm.png}
    \caption{Weighted Score Model}
    \label{fig:wsm}
\end{figure}

The chart shown in Figure \ref{fig:wsm} provides a visual representation of the model.

\subsection{Financial Analysis}

Before reaching a final conclusion on project selection, it is crucial to conduct a comprehensive financial analysis to ensure maximum benefit to our client. Our team performed an analysis on the Net Present Value (NPV) and Return on Investment (ROI) to verify that the selected project aligns with financial objectives and delivers optimal returns.

\subsubsection*{Net Present Value (NPV)}

Net Present Value (NPV) is a financial metric used to evaluate the profitability of an investment or project. It represents the difference between the present value of cash inflows and the present value of cash outflows over the investment's lifetime.

The Net Present Value (NPV) is calculated using the following formula:

\[
\text{NPV} = \sum_{t=0}^{n} \frac{C_t}{(1 + r)^t}
\]

Where:
\begin{itemize}
    \item \( C_t \) represents the cash flow at time \( t \),
    \item \( r \) is the discount rate,
    \item \( t \) is the time period (usually in years),
    \item \( n \) is the total number of periods.
\end{itemize}

This formula discounts each of the cash flows back to their present value and then sums them up. NPV helps in assessing the profitability of a project by comparing the present value of cash inflows to the present value of cash outflows over the project's lifetime. A positive NPV indicates that the projected earnings (in present value terms) exceed the anticipated costs, thus making it a potentially profitable investment.



\section{Business Case Study}

